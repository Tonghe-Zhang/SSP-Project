
%Choose a conveniently small page size
\documentclass{article}
%\usepackage[paperheight=16cm,paperwidth=12cm,textwidth=10cm]{geometry}
\usepackage{lipsum}% for some dummy text

%package for headings
\setlength{\headheight}{20.60818pt}
\usepackage{ctex}
\usepackage{authblk} 
\usepackage{amsmath, amssymb, amsthm}
\usepackage{moreenum}
\usepackage{mathtools}
\usepackage{url}
\usepackage{bm}
\usepackage{enumitem}
\usepackage{graphicx}
\usepackage{listings}
\usepackage{color}
\usepackage{subcaption}
\usepackage{booktabs} % toprule
\usepackage[mathcal]{eucal}
%top of a page
% \usepackage{fancyhdr}
% \pagestyle{fancy}
% \fancyhead[L]{\runtitle}
% \fancyhead[R]{\runauthor}

%necessary headers for maths/physics
\usepackage{physics}
\usepackage{amsfonts}
\usepackage{amsmath}
\usepackage{pxfonts}
\usepackage{amsfonts, amssymb}

%package for insertion of figures and tables
\usepackage{graphicx}
\usepackage{float}
\usepackage[export]{adjustbox} 
\usepackage{caption}
\usepackage{subcaption}
\captionsetup[figure]{name={fig }}
\captionsetup[table]{name={table }}

%package for references,annotations and captions 
    \usepackage[justification=centering]{caption} %Manually make all the captions stay in middle of the line.

    \usepackage{pdfcomment}

    \newcommand{\commentontext}[2]{\colorbox{yellow!60}{#1}\pdfcomment[color={0.234 0.867 0.211},hoffset=-6pt,voffset=10pt,opacity=0.5]{#2}}
    \newcommand{\commentatside}[1]{\pdfcomment[color={0.045 0.278 0.643},icon=Note]{#1}}
    \newcommand{\todo}[1]{\commentatside{#1}}
    \newcommand{\TODO}[1]{\commentatside{#1}}
    
    %\usepackage{hyperref}  % 
        \hypersetup{hidelinks,
        colorlinks=true,
        allcolors=black,
        pdfstartview=Fit,
        breaklinks=true
    }

%package to insert codes
\usepackage{algorithm}
\usepackage{algorithmic}
\usepackage{color}
\usepackage{listings}
\lstset{ %
language=C++,                % choose the language of the code
basicstyle=\footnotesize,       % the size of the fonts that are used for the code
numbers=left,                   % where to put the line-numbers
numberstyle=\footnotesize,      % the size of the fonts that are used for the line-numbers
stepnumber=1,                   % the step between two line-numbers. If it is 1 each line will be numbered
numbersep=5pt,                  % how far the line-numbers are from the code
backgroundcolor=\color{white},  % choose the background color. You must add \usepackage{color}
showspaces=false,               % show spaces adding particular underscores
showstringspaces=false,         % underline spaces within strings
showtabs=false,                 % show tabs within strings adding particular underscores
frame=single,           % adds a frame around the code
tabsize=2,          % sets default tabsize to 2 spaces
captionpos=b,           % sets the caption-position to bottom
breaklines=true,        % sets automatic line breaking
breakatwhitespace=false,    % sets if automatic breaks should only happen at whitespace
escapeinside={\%*}{*)}          % if you want to add a comment within your code
}

%codes settings
\lstset{
    basicstyle          =   \sffamily,         
    keywordstyle        =   \bfseries,        
    commentstyle        =   \rmfamily\itshape,  % annotations/comments are in italian style
    stringstyle         =   \ttfamily, 
    flexiblecolumns,                
    numbers             =   left,   % 行号的位置在左边
    showspaces          =   false,  % 是否显示空格,显示了有点乱,所以不现实了
    numberstyle         =   \zihao{-5}\ttfamily,    % 行号的样式,小五号,tt等宽字体
    showstringspaces    =   false,
    captionpos          =   t,      % 这段代码的名字所呈现的位置,t指的是top上面
    frame               =   lrtb,   % show the frame
}

\lstdefinestyle{Python}{
    language        =   Python,
    basicstyle      =   \zihao{-5}\ttfamily,
    numberstyle     =   \zihao{-5}\ttfamily,
    keywordstyle    =   \color{blue},
    keywordstyle    =   [2] \color{teal},
    stringstyle     =   \color{magenta},
    commentstyle    =   \color{red}\ttfamily,
    breaklines      =   true,   % 自动换行,建议不要写太长的行
    columns         =   fixed, 
    basewidth       =   0.5em,
}


% Choose a conveniently small page size
\title{\textbf{Signal and System \\ Course Project Report}}
\author{张同和\\ \href{zhang-th21@mails.tsinghua.edu.cn}{zhang-th21@mails.tsinghua.edu.cn}}
\affil{Dep.EE,Tsinghua University}



\makeatletter
\let\runauthor\@author
\let\runtitle\@title
\makeatother

\begin{document}
\maketitle


%table of contents
\renewcommand\contentsname{Table of Contents}
\tableofcontents

%main content starts from here.
\newpage

\part{\textbf{Verification of correlation property}}

\section{Algorithm design for task 1}
\par In this task, we wish to verify the following properties:
\begin{equation}
    \forall i \in \{0,...m-1\},\tau \in \{-2N,...2N\}:\\
    \sum_{k=0}^{N-1} c_i[k]c_i[k-\tau]=Q_{i}[\tau]
    \label{formula_auto}
\end{equation}

\begin{equation}
    \forall i \ne j, i,j \in \{0,...m-1\},\tau \in \{-2N,...2N\}:\\
    \sum_{k=0}^{N-1}c_{i}[k]c_j[k-\tau] =R_{i,j}[\tau]=0,
    \label{formula_corr}
\end{equation}

\par where $m=63$ is the number of satellites and 
$\{c_i[k]\}_{i=0}^{i=m-1}$ are a family of signals with length $N=f_s\cdot T=51.15MHZ\cdot 1ms=51150$.

\par Since the equations are about correlation and autocorrelation, we rewrite them in the form 
of convolution and apply Fourier transform to both sides, we find that:

\begin{equation}
    Q_{i}[k]=\mathbf{DFT^{-1}} [\mathbf{DFT}[c_i[k]]\cdot \mathbf{DFT}[c_i[-k]]=\mathbf{DFT}^{-1}[\abs{X_i[k]}^2] 
    \label{formula_fourier_analysis_auto}
\end{equation}

\begin{equation}
    R_{i,j}[k]=\mathbf{DFT^{-1}} [\mathbf{DFT}[c_i[k]]\cdot \mathbf{DFT}[c_j[-k]]=\mathbf{DFT}^{-1}[X_i[k]\cdot X_j^{*}[k]] 
    \label{formula_fourier_analysis_corr}
\end{equation}



\par
The convolution between two finite-length discrete signals a and b is defined as the convolution of 
two infinite sequences generated by zero-padding the finite signals, then restricting it to its support.
If the finite signals a and b have the length N and M respectively, then their convolution has a finite length of N+M-1.
\\ \indent
\textbf{This conclusion shows that only 2N-1 elements in the correlation and autocorrelation results of $\{c_i[k]\}_{k=0}^{k=N-1}$ need not be zero, which 
assumes that the length of the output sequence is 2N-1.}
To extend the result to $\{-2N,...2N\}$ as the task description requires, 
we have to zero-pad the convolution result.
\\The following analysis aims to determine the places to zero-pad.

\begin{equation}
    \tilde{a}[k]=
    \begin{cases}
        a[k],& k\in \{0,...N-1\} \\
        0,&else
    \end{cases}
\end{equation}

\begin{equation}
    \tilde{b}[k]=
    \begin{cases}
        b[k],& k\in \{0,...N-1\} \\
        0,&else
    \end{cases}
\end{equation}

\begin{equation}
    corr(a,b)[\tau]=R_{ab}[\tau]\sum_{i=0}^{N-1}\tilde{a}[k]\cdot \tilde{b}[k-\tau]
\end{equation}

\begin{equation}
    if \ \forall k \in \{0,N-1\}: a[k]b[k-\tau]=0, \ then\ R_{a,b}[\tau]=0 
\end{equation}

\begin{equation}
    if \ \forall k \in \{0,N-1\}: k-\tau \leq -1\  or \ k-\tau \geq N\ then\ R_{a,b}[\tau]=0 
\end{equation}

\begin{equation}
    if \ \forall k \in \{0,N-1\}: \tau \notin (-N,N) \ then\ R_{a,b}[\tau]=0 
\end{equation}

\begin{equation}
    R_{ab}[\tau]=
    \begin{cases}
        R_{ab}[\tau],& k\in \{-(N-1),...(N-1)\} \\
        0,&else
    \end{cases}
\end{equation}

\indent Eventually, we have to zero-pad N+1 elements on both the left and right borders of the original domain, which makes the sequence own a length of 4N+1.

\begin{equation}
    R_{ab}[\tau]=
    \begin{cases}
        R_{ab}[\tau],& k\in \{-(N-1),...(N-1)\} \\
        0,& k \in \{-2N,...-N\} \cup \{N,...2N\}\\
        undefined,& else
    \end{cases}
\end{equation}

Unfortunately, the DFT algorithm does not place elements with negative indices on the left of the positive
domain, instead, it puts Q[-2N:-1] in the place of Q[2N+1:4N].To solve this problem, we first append 3N+1 zeros to 
$c_i[k]$, then apply DFT and $DFT^{-1}$ before moving Q[2N+1:4N] back to Q[-2N:-1]. \textbf{This slicing operation
is referred to as method 'curl' in the source codes.}

\section{Experiment results of task 1}
\subsection{Autocorrelation Analysis}

\par To verify autocorrelation property \ref{formula_auto}, we execute the algorithm defined above and plot the output sequences.
\begin{figure}[H]
\includegraphics[width=\textwidth,height=0.5\textwidth]{auto_24.png}
\end{figure}
\begin{figure}[H]
\includegraphics[width=\textwidth,height=0.5\textwidth]{auto_33.png}
\end{figure}
\begin{figure}[H]
\includegraphics[width=\textwidth,height=0.5\textwidth]{auto_58.png}
\caption{Autocorrelation of c24,c33 and c58}
\end{figure}

\textbf{The graphs show that autocorrelation reaches its maximum at $\tau=0$, which fits theroretical properties of autocorrelation.
Furthermore, the maximum value of autocorrelation is exactly 51150=N, which is supported by the fact that experimental data of 
$c_i[k]$ is a series with values oscillating between -1 and 1.}


\subsection{Correlation Analysis}
\par To verify correlation property \ref{formula_corr}, we randomly choose two different signals  $c_{i}[k]$ and $c_{j}[k]$
and calculate their correlation.Then we compare the result with the autocorrelation of $c_i[k]$.
\begin{figure}[H]
    \centering
    \begin{subfigure}[b]{\textwidth}
        \centering
        \includegraphics[width=\textwidth]{corr_24_20.png}
        \caption{$corr(c_{24},c_{24})\ and\ corr(c_{24}, c_{20})$}
    \end{subfigure}
    \hfill
    \begin{subfigure}[b]{\textwidth}
        \centering
        \includegraphics[width=\textwidth]{corr_33_13.png}
        \caption{$corr(c_{33},c_{33})\ and\ corr(c_{33}, c_{13})$}
    \end{subfigure}
    \begin{subfigure}[b]{\textwidth}
        \centering
        \includegraphics[width=\textwidth]{corr_58_6.png}
        \caption{$corr(c_{58},c_{58})\ and\ corr(c_{58}, c_{6})$}
    \end{subfigure}
    \caption{Correlation Analysis compared with autocorrelation}
    \label{fig_corr}
\end{figure}

\textbf{Although the correlation function is not strictly zero, its magnitude $max|R_{i,j}[k]|$ with respect to the autocorrelation of
the first signal $max\abs{c_{i}[k]}$  is sufficiently small. 
This shows that the correlation between different signals can be treated as zero.
Detailed analysis of their statistics attributes is listed in Table \ref{tab_corr}. (The column "relative mag" 
stands for $\frac{max|R_{i,j}[k]|}{max\abs{c_{i}[k]}}$.)}

\begin{table}[H]
    \centering
    \caption{Correlation Analysis}
    \begin{tabular}{c||c c c c}
        correlation pair & mean & std &range &relative mag\\
        \hline
        $c_{24}$ and $c_{20}$ & -2.328 &208.932 & 2984.000 & 2.917\%\\
        $c_{33}$ and $c_{13}$ & 0.332 &205.931  & 2788.000 & 2.725\%\\
        $c_{58}$ and $c_{6}$ & 0.000 &209.516  &2873.000 & 2.808\%\\
    \end{tabular}
    \label{tab_corr}
\end{table}

\part{\textbf{Acquisition of satellites}}

\section{Algorithm design for task 2}

\par We try to utilize the correlation property between $c_i$ to pick the satellites from r[k].
\par If a signal y(t) that contains white noise w(t) is generated by 
scaling and shifting another signal x(t), then we can find the \
phase shift by simply calculating their correlation.

\begin{eqnarray}
    corr(Ax(t-t_0)+w(t),x) &= &A\cdot corr(x(t-t_0),x)+corr(w,x) \nonumber    \\
    &=& A\cdot \int_{\infty}^{\infty} x(t-t_0)x(t-\tau)dt+R_{w,x}(\tau) \nonumber    \\
    &=& A\cdot \int_{\infty}^{\infty} x(t-(t_0-\tau))x(t)dt+R_{w,x}(\tau) \nonumber    \\
    &=& A\cdot R_{xx}(t_0-\tau)+R_{w,x}(\tau)
\end{eqnarray}

\par Since w is white noise, we suppose that it is not related to x(t) and thus $R_{w,x}(\tau)=0$, which 
yields $\mathbf{corr(y,x)(\tau)=A\cdot R_{xx}(t_0-\tau)=A\cdot R_{xx}(\tau-t_0)}$. 
\par \indent Considering the fact that autocorrelation $R_{xx}(\tau)$ reaches its maximum when $\tau=0$, we can 
get the value of $t_0$ by \textbf{setting $t_0=argmax(corr(x,y))$.}

\par The data at hand $r[k]$ is a mixture of periodic signals with length=1000T.
\\After restricting r[k] to a single period, if it contains a component generated by shifting $c_i$ by $\tau_i$, 
then its correlation with $c_i$ will have a spike at $\tau=\tau_i$.
Otherwise, the correlation will be a plain signal with no "spikes", because all the components of r[k] are
not related with $c_i[k]$.(see formula \ref{formula_corr}).

\begin{eqnarray}
    corr(r,c_i)[\tau]\vert_T&=&corr(\sum_{j\in V} (s_j(kT_s-\tau_j)+w_j[k])\vert_T,c_i)    \nonumber    \\
    &=&\sum_{j\in V} corr(s_j(kT_s-\tau_j)\vert_T+w_j[k]\vert_T,c_i)    \nonumber    \\
    &=&\sum_{j\in V} corr(c_j(kT_s-\tau_j)+w_j[k]\vert_T,c_i)                                    \nonumber    \\
    &=&\mathbf{1}(i\in V) R_{c_i,c_i}(\tau-\tau_i)
\end{eqnarray}

\textbf{To test the existence of spikes with a quantitative approach, we set $max(corr(r,c_i))\geq 5\times 10^3$ as the indicator of spikes.}
With the exact structure of $\{c_i[k]\}_{i=0}^{N-1}$ obtained from task 1, 
we can design the satellite discovery algorithm as follows:

\begin{algorithm}
    \caption{Find Satellites}
    \label{algo_find_sat}
    \begin{algorithmic} 
    \STATE $N \ \leftarrow 51150$
    \STATE $M \ \leftarrow 5000$
    \STATE $r \ \leftarrow r\vert_{[0,N)}$
    \FOR {i=0,1,\ldots,m-1}
        \STATE $corr=correlate(r[k],c_i[k])$
        \IF {max(corr)$\geq$ M}
            \STATE $\tau_i \leftarrow argmax(corr)\%N$ 
        \ENDIF
    \ENDFOR
    \STATE $V \leftarrow len(\{\tau_i\})$
    \RETURN $V,\{\tau_i\}$
\end{algorithmic}
\end{algorithm}

\par \textbf{We should note that in the algorithm, we must compute $corr(r,c_i)$, but not $corr(c_i,r)$.
\\\indent That is because $\tau_i=argmax(corr(r,c_i))=N+1\ - \ argmax(corr(c_i,r)) \ne argmax(corr(c_i,r))$.}

\section{Experiment results of task 2}
\par Implementation of the algorithm \ref{algo_find_sat} allows the author to \textbf{find four satellites.}
\par During the execution of the algorithm, the console outputs progress information and pauses to display correlation results
between signals of discovered satellites and r[k].

\begin{figure}[H]
\includegraphics[width=\textwidth,height=0.3\textwidth]{find_satellite_2.jpg}
\caption{Running result of task 2}
\end{figure}

\par The time-domain correlation graphs of the acquired satellites are shown below.
\begin{center}
\begin{figure}[H]
    \centering
    \begin{subfigure}[H]{0.45\textwidth}
        \centering
        \includegraphics[width=\textwidth]{sat_0.png}
        \caption{$C_0$}
        \end{subfigure}
    \begin{subfigure}[H]{0.45\textwidth}
        \centering
        \includegraphics[width=\textwidth]{sat_2.png}
        \caption{$C_2$}
    \end{subfigure}

    % leave a blank line here to force tex to put images in another row.
    \begin{subfigure}[H]{0.45\textwidth}
        \centering
        \includegraphics[width=\textwidth]{sat_4.png}
        \caption{$C_4$}
    \end{subfigure}
    \begin{subfigure}[H]{0.45\textwidth}
        \centering
        \includegraphics[width=\textwidth]{sat_6.png}
        \caption{$C_6$}
    \end{subfigure}
    \centering
    \caption{Discovery of satellite signals from r[k].\\The Red lines stand for magnitude=5e3.}
    
    \label{find_satellites}
\end{figure}
\end{center}
\par \indent The author \textbf{found four satellites.}
\par \indent Detailed phaseshift information is recorded automatically by the program and written to a Txt file.
The information can be summarized as follows:

\begin{table}[H]
    \centering
    \begin{tabular}{c|c}
        i & $\tau_i$\\
        \hline
        1 & 5000\\
        3 & 7255\\
        5 & 22705\\
        7 & 2370\\
    \end{tabular}
    \caption{Phase shift information of the satellites acquired}
    \label{tab_compare_all}
\end{table}

\end{document}

% $\begin{bmatrix} 
%    a_{11} & a_{12} \\ a_{21} & a_{22}
% \end{bmatrix}$


% \begin{table}[H]
%     \centering
%     \begin{tabular}{c c}
%         i & $\tau_i$\\
%         \hline
%         0 & 46151\\
%         2 & 43896\\
%         4 & 28446\\
%         6 & 48781\\
%     \end{tabular}
%     \caption{Phase shift information of the satellites acquired}
%     \label{tab_compare_all}
% \end{table}



%https://www.overleaf.com/learn/latex/How_to_Write_a_Thesis_in_LaTeX_(Part_3)%3A_Figures%2C_Subfigures_and_Tables
% \begin{figure}[H]
%     \centering
%     \begin{subfigure}[H]{0.45\textwidth}
%         \centering
%         \includegraphics[width=\textwidth]{sat_0.png}
%         \caption{$C_0$}
%         \end{subfigure}
%     \begin{subfigure}[H]{0.45\textwidth}
%         \centering
%         \includegraphics[width=\textwidth]{sat_2.png}
%         \caption{$C_2$}
%     \end{subfigure}

%     % leave a blank line here to force tex to put images in another row.
%     \begin{subfigure}[H]{0.45\textwidth}
%         \centering
%         \includegraphics[width=\textwidth]{sat_4.png}
%         \caption{$C_4$}
%     \end{subfigure}
%     \begin{subfigure}[H]{0.45\textwidth}
%         \centering
%         \includegraphics[width=\textwidth]{sat_6.png}
%         \caption{$C_6$}
%     \end{subfigure}
%     \caption{Discovery of satellite signals from r[k].\\Red line suggests magnitude=5e3.}
    
%     \label{find_satellites}
% \end{figure}

% \begin{algorithm}
%     \caption{Find Satellites}
%     \label{algo_find_sat}
%     \begin{algorithmic} 
%     \STATE $N \ \leftarrow 51150$
%     \STATE $M \ \leftarrow 5000$
%     \STATE $r \ \leftarrow r\vert_{[0,N)}$
%     \FOR {i=0,1,\ldots,m-1}
%         \STATE $corr=correlate(r[k],c_i[k])$
%         \IF {max(corr)$\geq$ M}
%             \STATE $\tau_i \leftarrow argmax(corr)\%N$ 
%         \ENDIF
%     \ENDFOR
%     \STATE $V \leftarrow len(\{\tau_i\})$
%     \RETURN $V,\{\tau_i\}$
% \end{algorithmic}
% \end{algorithm}


%insert source codes..
%
%\begin{lstlisting}
%    loss = CrossEntropyLoss(pred, label )
%    loss.backward()
%    optimizer.zero_grad()
%    optimizer.step()
%\end{lstlisting} 